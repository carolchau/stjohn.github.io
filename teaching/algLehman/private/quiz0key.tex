\documentclass[11pt]{article}
\usepackage{graphicx}
\usepackage{amssymb}

\textwidth = 6.5 in
\textheight = 9 in
\oddsidemargin = 0.0 in
\evensidemargin = 0.0 in
\topmargin = 0.0 in
\headheight = 0.0 in
\headsep = 0.0 in
\parskip = 0.2in
\parindent = 0.0in

\newtheorem{theorem}{Theorem}
\newtheorem{corollary}[theorem]{Corollary}
\newtheorem{definition}{Definition}

\begin{document}
\begin{center}
	{Answer Key for Quiz 0}\\
	{CMP 761: Analysis of Algorithms}\\
	{3 September 2002}
\end{center}



Name: \underline{\hspace{ 3 in}}

Student ID (Social Security Number): \underline{\hspace{1in}}

\bigskip
Write your answer to each on a separate piece of paper.  Staple your
answer sheets to this sheet when you turn in the quiz.
\bigskip
\begin{enumerate}
  \item Prove for all natural numbers $n$ that:
	\[
		\sum_{i=1}^{n} i = \frac{n(n+1)}{2}
	\]
\bigskip
{\tt
  This can be done in several different ways.  One way to prove this is by 
  induction on $n$:

  \underline{Base Case:} $n=1$.\\
     When $n =1$, the left hand side 
     evaluates to 1 as does the right hand side ($\frac{1(1+1)}{2} = 1$).
     So, the equation holds for $n=1$.

  \underline{Inductive Step:} $n > 1$.\\
     Assume true for $n$, and then show true for $n+1$.
     Starting with the left hand side:
	$$
	\begin{array}{rcll}
		\sum_{i=1}^{n+1} i & = & (n+1) + \sum_{i=1}^{n} i &\\
			& = & (n+1)  + \frac{n(n+1)}{2} & \mbox{(by IH)}\\
			& = & \frac{2(n+1)}{2} + \frac{n(n+1)}{2} & \\
			& = & \frac{2(n+1)+n(n+1)}{2} & \\
			& = & \frac{2n+2+n^2+n)}{2} & \\
			& = & \frac{n^2 + 3n + 2}{2} & \\
			& = & \frac{(n+1)(n+2)}{2} & \\
			& = & \frac{(n+1)((n+1)+1)}{2} & \\
	\end{array}
	$$
  So, the equation holds for $n+1$, assuming it holds for $n$.

  Thus, by the principle of induction, for all natural numbers $n$,
	\[
		\sum_{i=1}^{n} i = \frac{n(n+1)}{2}
	\]
}

\bigskip
	
  \item Write (in pseudo-code) an algorithm that sorts a list of $n$
	numbers.
	
	\bigskip
	(More formally, design an algorithm for the following:\\
	{\bf Input:} A sequence of $n$ numbers $\{A[1],A[2],\ldots,A[n]\}$.\\
	{\bf Output:} A reordering $\{A'[1],A'[2],\ldots,A'[n]\}$
		of the input sequence such that
		$A'[1] \leq A'[2] \leq \cdots \leq A'[n]$.)
\end{enumerate}

{\tt Again, there are many different answers to this question.  One possible
way to sort the list would be a bubble sort:

\begin{verbatim}
  for (i = 0; i < n; i++)
  {
    for (j=0; j < n-1; j++)
    {
      if (A[j] > A[j+1])
      {
         /* Swap the two elements */
        tmp = A[j];
        A[j] = A[j+1];
        A[j+1] = tmp;
      }
    }
  }
\end{verbatim}

}

 \end{document}