% Exam 1 for Coen 251
% 13 April 1998

\documentstyle[12pt]{article}
\pagestyle{myheadings}
\markboth{Name:\underline{\hspace{3in}}}
{Name:\underline{\hspace{3in}}}
\textwidth=6 true in
\textheight=8.5 true in
\oddsidemargin = 0.0 true in
\evensidemargin = 0.0 true in
\newcommand{\ul}{\underline}
\newcommand{\spa}{\hspace{.25in}}
\begin{document}

{\Large
\begin{center}
\mbox{ }
\vspace{.1in} \\
    Exam 1 \\
    Computer Science 751\\ 
    Lehman College-- CUNY\\ 
    Thursday, 17 October 2002
\end{center}
}

{\large
\begin{center}
\begin{tabular}{ll}
\mbox{ }
\vspace{.5in} \\
NAME (Printed) & \ul{\hspace{3in}}\\ 
NAME (Signed) & \ul{\hspace{3in}} \\
Login & \ul{\hspace{3in}}\\
%Day of Lab Session & \ul{\hspace{3in}}\\
\end{tabular}
\end{center}
}

Please show all your work and circle your answers.  Your grade 
will be based on the the work shown.  
{\large
\begin{center}
\mbox{ }
\vspace{.5in} \\
\begin{tabular}{|c|l|l|}
\hline \hline
\hspace{.05in} Question 1 \hspace{.05in} & \hspace{.5in} \mbox{ } & 10 points \\ \hline
\hspace{.05in} Question 2 \hspace{.05in} & \hspace{.5in} \mbox{ } & 15 points \\ \hline
\hspace{.05in} Question 3 \hspace{.05in} & \hspace{.5in} \mbox{ } & 15 points \\ \hline
\hspace{.05in} Question 4 \hspace{.05in} & \hspace{.5in} \mbox{ } & 10 points \\ \hline
\hspace{.05in} Question 5 \hspace{.05in} & \hspace{.5in} \mbox{ } & 15 points \\ \hline 
\hspace{.05in} Question 6 \hspace{.05in} & \hspace{.5in} \mbox{ } &15 points \\ \hline
\hspace{.05in} Question 7 \hspace{.05in} & \hspace{.5in} \mbox{ } &20 points \\ \hline

\hspace{.05in} TOTAL \hspace{.05in} & \hspace{1in} \mbox{ }  &100 points \\ \hline \hline
\end{tabular}
\end{center}
}

\newpage
\begin{center}
\begin{large}
Useful Formulas
\end{large}
\end{center}

$$
\begin{array}{lr}
	\sum_{i=1}^n i = \frac{n(n+1)}{2}
	& 
	\sum_{i=1}^n x^i = \frac{x^{n+1} - 1}{x-1}
	\\
	\\
	\sum_{i=1}^n \frac{1}{i} = \ln n + O(1)
	& 
	\sum_{i=0}^{\infty} ix^i = \frac{x}{(1-x)^2}
	\\
	\\
	e^x = 1 + x + \frac{x^2}{2!} + \frac{x^3}{3!} + \ldots
	& 
	\lim_{n\rightarrow\infty} (1+\frac{x}{n})^n = e^x
	\\
	\\
	n! = \sqrt{2\pi n} (\frac{n}{e})^n (1 + \Theta(\frac{1}{n}))
	& 
	n! = o(n^n)
	\\
	\\
	n! = \omega(2^n)
	& 
	\lg(n!) = \Theta(n\lg n)
	\\
	\\
	{n \choose k} = \frac{n!}{k!(n-k)!}
	& 
	(\frac{n}{k})^k \leq {n \choose k} \leq (\frac{en}{k})^k
	\\
	\\
	E[X] = \sum_x x Pr[X=x]
	& 
	Var[X] = E[(X-E[X])^2]\\ 
	&= E[X^2] - E^2[X]

\end{array}
$$

\newpage
\begin{enumerate}

%Question 1:  5 T/F
    \item True or False (2 point each):
        \begin{enumerate}
    	    \item \underline{\hspace{.25in}} $\lg n = o(n^2)$.
    	    \item \underline{\hspace{.25in}} $\lg n = O(n^2)$.   
    	    \item \underline{\hspace{.25in}} $3n^2 + 2n = \omega(n)$.
    	    \item \underline{\hspace{.25in}} $3n^2 + 2n = \Omega(n)$.  

    	    \item \underline{\hspace{.25in}} $n! = \Omega(2^n)$.	        	    \item \underline{\hspace{.25in}} $\lg(n!) = \Theta(\lg(n^n))$.	 	    	 
    	     \item \underline{\hspace{.25in}} $f(n) = o(g(n))$ implies 
    	    	$f(n) = O(g(n))$.
    	    \item \underline{\hspace{.25in}} $f(n) = \Omega(g(n))$ implies 
    	    	$f(n) = \Theta(g(n))$.
    	    \item \underline{\hspace{.25in}} $f(n) = \Theta(g(n))$ implies 
    	    	$f(n) = O(g(n))$.
    	    \item \underline{\hspace{.25in}} $f(n) = \Theta(g(n))$ implies 
    	    	$f(n) = \omega(g(n))$.
    	    		    	 
    	\end{enumerate}
%\newpage
%Question 2: running time of 2 simple functions, bucket, radix, counting,
		%partition?
	\item Assume that every statement takes a constant $c$ time.  Give
		tight bounds on the order of growth and justify your answer:
		\begin{enumerate}
%			\item Assume $L$ is a singly linked list and $k$ is a key:
%\begin{verbatim}
%LIST-SEARCH(L,k)
%1  x <- head[L]
%2  while x != NIL and key[x] != k
%3       do x <- next[x]
%4  return x
%\end{verbatim}
%			\item Assume $A$ is an array storing a heap and $k$ is a key:
%\begin{verbatim}
%HEAP-INSERT(A,k)
%1  heap-size[A] <- heap-size[A] + 1
%2  i <- heap-size[A]
%3  while i > 1 and A[PARENT(i)] < key
%4       do A[i] <- A[PARENT(i)]
%5          i <- PARENT(i)
%6  A[i] <- key
%\end{verbatim}
    \item What is the output, assuming the following piece of code
      is embedded in a complete and correct program:
\begin{verbatim}
for ( int i = 5; i > 0; i--)
{
   for ( int j = 0 ; j < i; j++)
       cout << '*';
   cout << endl;
}
\end{verbatim}
\bigskip
			\item Assume A is an array of length $n$:
\begin{verbatim}
FIND-MAX(A)
1  max <- - infinity
2  for i <- 1 to n
3       do if A[i] > max
4          then max <- A[i]
5  return max
\end{verbatim}
\bigskip
			\item Assume A is an array and the function COMBINE takes $\Theta(n)$
				on a sublists {\tt A[p..r]} and {\tt A[r+1..p]} of combined length $n$:
\begin{verbatim}
MSORT(A,p,q)
1  if ( q - p > 1)
2       do MSORT(A,p, q/2);
3            MSORT(A,q/2+1,q);
4            COMBINE(A,p,q/2,r);
\end{verbatim}

		\end{enumerate}
\newpage

\item Assume $A$ is an array storing a heap and $k$ is a key:
\begin{verbatim}
HEAP-INSERT(A,k)
1  heap-size[A] <- heap-size[A] + 1
2  i <- heap-size[A]
3  while i > 1 and A[PARENT(i)] < key
4       do A[i] <- A[PARENT(i)]
5          i <- PARENT(i)
6  A[i] <- key
\end{verbatim}

    	\begin{enumerate}
		\item What does the heap look like 
			inserting keys from the sequence: $\{10,3,1,12,20,18,14,16\}$?
			\vspace{2in}
		\item What is the height of the heap
				from inserting keys from the sequence: $\{10,3,1,12,20,18,14,16\}$?
				\\ \\
				\ \hfill\underline{\hspace{1in}}
		\item Write a function that will take a heap (stored in an array called A)
			and return the maximum value.
			\vspace{2in}
    	\end{enumerate}
\newpage
%Question 5: 2 simple recurrences
	\item Give asymptotic upper and lower bounds for $T(n)$ for the 
		following two recurrences.  Make your bounds as tight as possible,
		and justify your answers:\\
		Assume that $T(n)$ is constant for $n\leq 2$:
		\begin{enumerate}
			\item $T(n) = 5T(n/3) + 1$		
				\vspace{2in}
			\item $T(n) = 10T(n-2) + n$
				\vspace{2in}
		\end{enumerate}

\newpage    	
%Question 7: write code for max/min/search?
	\item Assume $A[1..n]$ is an array.  
\begin{verbatim}
FIND-KEY(A,k)
1  for i <- 1 to n
2       do if A[i] = k
3          then return i
\end{verbatim}
		\begin{enumerate}
			\item What are tight bounds on the {\bf worst case} order of growth?
				Justify your answer:
				\vspace{1.5in}
			\item What are tight bounds on the {\bf best case} order of growth?
				Justify your answer:
				\vspace{1.5in}
			\item What are tight bounds on the {\bf average case} order of growth,
				assuming that all numbers in A are randomly drawn from the 
				interval $[1,n]$?
				Justify your answer:
				\vspace{1.5in}
		\end{enumerate}
\newpage

	\item Suppose that we have an array of $n$ objects to sort and that
		the key of each record has the value $\{0,1,\ldots,k\}$. Assume that
		$k$ is much smaller than $n$ ($k = o(n)$).   Give a simple,
		{\bf linear-time} algorithm for sorting the $n$-objects.
		\vspace{3in}
		
	\item Suppose that we have an array of $n$ objects to sort, and there
		are no conditions on the keys.  
		\begin{enumerate}
			\item What is the lower bound on the worst case running time of 
				a comparison sort of the array A?
				\vspace{.75in}
			\item Write a sorting algorithm that sorts a list with the worst case
				running time you stated above:
				\vspace{3in}
		\end{enumerate}

		
\end{enumerate}
\end{document}



