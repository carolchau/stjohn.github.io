\documentstyle[12pt]{article}
\pagestyle{empty}
\textwidth=6.5 true in
\textheight=9 true in
\topmargin = 10pt
\oddsidemargin = 0.0 true in
\evensidemargin = 0.0 true in
\newcommand{\ul}{\underline}
\newcommand{\spa}{\hspace{.25in}}
\begin{document}

{\Large
\begin{center}
    Answer Key for Exam 3 \\
    Computer Science 125 \\
    Boise State University\\
    Friday, 16 April 1999
\end{center}
}


\begin{enumerate}

%1 True or False:
        \item True or False:
        \begin{enumerate}
            \item \underline{F} In Java, files can be opened
                for reading, or writing, but not both.

            \item \underline{T} Classes can have data fields that
                are arrays of values.

            \item \underline{F} Two dimensional arrays always 
		have the same number of rows and columns.

            \item \underline{T} All elements of an array are 
		of the same type.

            \item \underline{T} A Java source file can contain
                any number of classes with package access.

            \item \underline{F} Data fields can only be declared
                as {\tt private}.

            \item \underline{T} {\tt System.out} is a static
                public variable.

            \item \underline{T} {\tt Math.PI} is a static public variable.

            \item \underline{T} After the following code is 
		executed {\tt b[2]} has the value 3.
\begin{verbatim}
     int[] a = {1, 2, 3, 4, 5};
     int[] b = a;
     a[2] = 5;
\end{verbatim}


            \item \underline{T} The following code copies array {\tt a}
                into array {\tt b}, in reverse order:
\begin{verbatim}
    for (int i = 0; i < a.length; i++)
        b[i] = a[i];
\end{verbatim}

\end{enumerate}


%2 Short Answer
        \item 
            \begin{enumerate}
            	\item Write code that sets up an array {\tt cheese} to hold 
		    10 strings.
		    
\begin{verbatim}
String [] cheese = new String[10];
\end{verbatim}		    
            	   
            	\item Write code that fills each element of the array 
                    {\tt cheese} with the string {\tt "cheddar"}.
		    
\begin{verbatim}
for (int i=0; i < 10; i++)
  cheese[i] = "cheddar";
\end{verbatim}		    
            	   
           	    
            	\item Write code that opens, for reading, a text 
		    file with the name, {\tt input.dat}.
		    
\begin{verbatim}
TextInputStream ins = new TextInputStream("input.dat");
\end{verbatim}		    
            	   

            	    
            	\item Write code that opens a text file with the name, 
                    {\tt output.dat}, for writing, and writes 
                    {\tt "Hello, world"} to the file {\tt output.dat}.
		    
\begin{verbatim}
TextOutputStream outs = new TextOutputStream("output.dat");
outs.println("Hello, world");
\end{verbatim}		    
            	   
            	    
            \end{enumerate}



%3 Draw $r = cos n *\Theta$ for n=2, n=5
        \item Consider the function:
\begin{verbatim}
public void rosie(int n)
{
   Point oldPt = new Point(5,0);
   Point newPt;

   for (int i = 0 ; i < 100 ; i++)
   {
      double theta = i*n*Math.PI/100;
      double r = 5*Math.cos(5*theta);
      newPt = new Point(r*Math.cos(theta), r*Math.sin(theta));
      new Line(oldPt, newPt).draw();
      oldPt = newPt; 
   }
}
\end{verbatim}

{\bf There is a typo in the above function.  The intended function
was:
\begin{verbatim}
public void rosie(int n)
{
   Point oldPt = new Point(5,0);
   Point newPt;

   for (int i = 0 ; i < 100 ; i++)
   {
      double theta = i*n*Math.PI/100;
      double r = 5*Math.cos(theta);
      newPt = new Point(r*Math.cos(theta), r*Math.sin(theta));
      new Line(oldPt, newPt).draw();
      oldPt = newPt; 
   }
}
\end{verbatim}
}

         \begin{enumerate}
             \item What is the output of {\tt rosie(2)}?\\	       	
        	{\bf Output:}
        	
INSERT JPG

        	{\bf Output of intended function:}
        	
INSERT JPG
        	
            \item What is the output of {\tt rosie(5)}?\\
        	{\bf Output:}
        	
INSERT JPG

        	{\bf Output of intended function:}
        	
INSERT JPG
        	
        	        	
                \end{enumerate}


%4
        \item Given the input file, {\tt winnie.hum}:


        	\begin{tabular}{|l|}
            	\hline
LINES WRITTEN BY A BEAR OF VERY LITTLE BRAIN \\
On Monday, when the sun is hot \\
I wonder to myself a lot: \\
"Now is it true, or is it not, \\
"That what is which and which is what?" \\
            	\hline
        	\end{tabular}

{\em (From work of A.~A.~Milne)}

	\begin{enumerate}

	    \item What is the output of the following program:
\begin{verbatim}
import ccj.*;
public class partA
{ public static void main(String[] args)
  { TextInputStream in = new TextInputStream("winnie.hum");
    TextOutputStream out = new TextOutputStream("humming");
    boolean alt = false;
    while (!in.fail()) {  
      String s = in.readLine();
      if (!in.fail()) {
        if ( alt ) {
          out.println(s);
          alt = !alt;
        }
        else {
          System.out.println(s);
          alt = !alt;
        }
      }
    }
  }
}
\end{verbatim}
        	\begin{tabular}{|l|}
		\hline
        	{\bf Screen Output:}\\
            	\hline
            	\\
LINES WRITTEN BY A BEAR OF VERY LITTLE BRAIN \\
I wonder to myself a lot: \\
"That what is which and which is what?" \\
            	\\
            	\hline
        	\end{tabular}

        	\begin{tabular}{|l|}
            	\hline
        	{\bf Contents of {\tt humming}:}\\
            	\hline
            	\\
On Monday, when the sun is hot \\
"Now is it true, or is it not, \\
            	\\
            	\hline
        	\end{tabular}

	    \item What is the output of the following program:
\begin{verbatim}
import ccj.*;
public class partB
{ public static void main(String[] args)
   { TextInputStream in = new TextInputStream("winnie.hum");
     TextOutputStream out = new TextOutputStream("humming");
     int count = 0;
     while (!in.fail()) {  
       String s = in.readLine();
       if (!in.fail()) {
         out.println(s+"\n");
         count++;
       }
     }
     System.out.println("Count is " + count);
  }
}
\end{verbatim}
        	\begin{tabular}{|l|}
            	\hline
        	{\bf Screen Output:}\\
            	\hline
            	\\
Count is 5 \\
            	\\
            	\hline
        	\end{tabular}

        	\begin{tabular}{|l|}
            	\hline
        	{\bf Contents of {\tt humming}:}\\
            	\hline
            	\\
\\            	
On Monday, when the sun is hot \\
\\
I wonder to myself a lot: \\
\\
"Now is it true, or is it not, \\
\\
"That what is which and which is what?" \\
\\
\\
            	\hline
        	\end{tabular}

	\end{enumerate}


%5
        \item 
            \begin{enumerate}
            	\item Label each method and member 
 		variable in the class 
		{\tt Faces} with {\bf C} if it
            	is a class method or variable, or {\bf I} if it is an 
                instance method or variable.
\begin{verbatim}
public class Faces {
  Faces() {                                     // INSTANCE
    center = new Point(0,0); 
    radius = 5;
    numFaces++;
  }
  Faces(Point p, int r) {                      // INSTANCE
    center = p.clone(); 
    radius = r;
    numFaces++;
  }
  public void move(double dx, double dy) {     // INSTANCE
    center.move(dx,dy);
  }
  public void draw() {                         // INSTANCE
    new Circle(center, r).draw();
    new Point(center.getX() - radius/2, 
              center.getY() + radius/2).draw();
    new Point(center.getX() + radius/2, 
              center.getY() + radius/2).draw();
    new Circle(new Point(center.getX(), 
                         center.getY() - radius/2), 
               radius/4).draw();
  }
  private Point center;                       // INSTANCE
  private Point radius;                       // INSTANCE
  private static int numFaces;                // CLASS
}
\end{verbatim}

		\item After the following code is run:
\begin{verbatim}
    Faces me = new Faces();
    Faces you = new Faces(new Point(3,3), 2);
    Faces him = new Faces();
\end{verbatim}
		
		    How many copies are there of the variable {\tt center}? 
		    \begin{tabular}{|c|}
		    \hline
		    3\\
		    \hline
		    \end{tabular}
		    
		    How many copies are there of the variable {\tt numFaces}?
		    \begin{tabular}{|c|}
		    \hline
		    1\\
		    \hline
		    \end{tabular}
		    
		    What is value of the variable {\tt numFaces}?
		    \begin{tabular}{|c|}
		    \hline
		    3\\
		    \hline
		    \end{tabular}
            \end{enumerate}


%6
        \item You are working as a programmer for a large company which 
		is shipping a new version 
        of its operating system next week.  While looking over the 
		code you have written 
        you suddenly notice that a section of the code which 
		handles error checking is "commented out". 
        Frantically you look over the test suits for the code, 
		which were created by an outside consultant, 
        and realize that none of them would have created the error 
		that the code was 
        supposed to handle.  The company has 10,000 boxes ready to ship.

		\begin{enumerate}
			\item What is the ethical dilemma that you face?

			\item What are {\bf three} possible ways of 
				dealing with the situation and 
				what are possible ramifications of each?

		\end{enumerate}


%7
        \item Write a {\bf complete applet} that plots the national budget
	   surplus/deficit for the last thirty years.  
	   The budget deficit numbers
           are stored in a file called {\tt budget.in}: 

        	\begin{tabular}{|l|}
            	\hline
            	1969 3\\
            	1970 -3\\
            	1971 -23\\
            	1972 -23\\
            	1973 -15\\
		\vdots\\
            	\hline
        	\end{tabular}

	   The first number on each line is the year, the second is the 
           budget surplus or deficit in billions of dollars.  You may 
           assume that the surplus/deficit is between than -300 to 100
           billion dollars.

           The output of your program should be a line graph of 
           the data from 
           the file {\tt budget.in}, where the x-axis is the year and
           the y-axis is the amount of the surplus/deficit. 

\begin{verbatim}
import ccj.*;

public class budget extends GraphicsApplet
{
  public void run()
  {
    setcoord(1969,100,1999,300);   //Resize to fit budget data
    
    TextInputStream in = new TextInputStream("budget.in");
                                   //Open data file for reading
    int year = 1969;               //The year in the data file
    int surp_def = in.readInt();   //Get first surplus/deficit
    Point newPoint = new Point(year,surp_def);//to set up initial point
    Point oldPoint;                //Used to save previous point
    
    while ( !in.fail() )
    {
      year++;            
      oldPoint = newPoint;
      surp_def = in.readInt();
      if ( !in.fail() )
      {
        newPoint = new Point(year, surp_def);
        new Line(oldPoint, newPoint).draw();
      }
    }
  }
}
\end{verbatim}          

%8
        \item The card game, Topdrops, is played on a deck of $n$
           cards with the cards numbered $1,2,\ldots,n$.  For example,
           if $n=5$, the cards are $1,2,3,4,5$.  The cards are laid
           face up on the table.  For example:
	   \begin{tabular}{|c|c|c|c|c|}
               \hline
               4&2&1&5&3\\
               \hline
           \end{tabular}

	   To play a round, you look at the number on the first card 
           (in this case: 4) and reverse that number of cards from
           left to right.  So, we reverse the first 4 cards to get:
	   \begin{tabular}{|c|c|c|c|c|}
               \hline
               5&1&2&4&3\\
               \hline
           \end{tabular}

	   The game stops when the card with $1$ reaches the far
           left.

           (The rest of the example game is:\\
	   \begin{tabular}{|c|c|c|c|c|}
               \hline
               3&4&2&1&5\\
               \hline
           \end{tabular}
	$\rightarrow$
	   \begin{tabular}{|c|c|c|c|c|}
               \hline
               2&4&3&1&5\\
               \hline
           \end{tabular}
	$\rightarrow$
	   \begin{tabular}{|c|c|c|c|c|}
               \hline
               4&2&3&1&5\\
               \hline
           \end{tabular}
	$\rightarrow$
	   \begin{tabular}{|c|c|c|c|c|}
               \hline
               1&3&2&4&5\\
               \hline
           \end{tabular}.)


      \begin{enumerate}
         \item
	   Write a recursive function that takes a hand 
           (an array of integers),
           prints the hand, and if the first card is not $1$, 
           plays a round and calls itself with the new hand.
\begin{verbatim}
public static void play(int[] hand)
{
  if ( hand.length() != 0)
  {
    //Print the hand:
    for ( int i=0 ; i < hand.length() ; i++)
      System.out.print(hand[i]+ " ");
    System.out.println();
    
    if ( hand[0] != 1 )
    {
      int j;
      int cardsToReverse = hand[0];  //Storage space to do reverse
      int[] temp = new int(CardsToReverse);
      
      //Reverse the cards:
      for ( j=0 ; j < CardsToReverse ; j++ )
        temp[i] = hand[i];
      for ( j=0 ; j < CardsToReverse ; j++ )
        hand[i] = temp[CardsToReverse - i - 1];
      
      //Call the function again:
      play(hand);      
    }
  }
}
\end{verbatim}

         \item  Write a program that prompts the user for the number
           of cards and the initial hand, and then calls your
           recursive function.

\begin{verbatim}
import ccj.*;

public class CardGame
{
  public static void main(String[] args)
  {
    System.out.print("Enter number of cards:");
    int numCards = Console.in.readInt();
    int[] hand = new int[numCards];
     
    for (int i = 0 ; i < numCards ; i++)
    {
      System.out.print("Enter card "+(i+1) + ": ");
      hand[i] = Console.in.readInt();
    }
    
    play(hand);
  }
}
\end{verbatim}


      \end{enumerate}


%9 Modify java and calculator programs:
        \item Clearly mark all changes:
                \begin{enumerate}
                        \item Add two buttons to the program below.  The
			    first button should have the label {\tt Black}
                            and change the drawing color to black.  The
	                    second button should have the label {\tt Purple}
                            and change the drawing color to purple.
\begin{verbatim}
// This example is modified from the book "Java in a Nutshell" 
// David Flanagan, O'Reilly & Associates, 1996.

import java.applet.*;	//Load in the applet class
import java.awt.*;	//Load in buttons and friends 

public class OneColorDraw extends Applet {
  private int last_x = 0;	//Keep track of x and y coordinate
  private int last_y = 0;	//When the mouse is clicked
  private Color current_color = Color.black;	//Current color
  private Button clear_button;	//Declare space for the clear button
  
  
//ADDED to store buttons:
private Button black_button;
private Button purple_button;
    
  // This function is called initially, when the program begins
  // executing.  It initializes the graphics window.
    
  public void init() {
    this.setBackground(Color.white);   //Set the background color
  
    // Create a button and add it to the graphics window
    clear_button = new Button("Clear");//Make label say "Clear"
    clear_button.setForeground(Color.black);
    clear_button.setBackground(Color.lightGray);
    this.add(clear_button);	//Necessary to use the button
    
//ADDED:
black_button = new Button("Black");//Make label say "Black"
black_button.setForeground(Color.black);
black_button.setBackground(Color.lightGray);
this.add(black_button);	//Necessary to use the button
purple_button = new Button("Purple");//Make label say "Purple"
purple_button.setForeground(Color.black);
purple_button.setBackground(Color.lightGray);
this.add(purple_button);	//Necessary to use the button
    
       
  }
    
  // Called when the user clicks the mouse to start a scribble
  // Sets the values of variables last_x and last_y to mark the
  // point of the click.
    
  public boolean mouseDown(Event e, int x, int y)
  {
    last_x = x; last_y = y;//Store coordinates when mouse clicked
    return true;	//and don't do anything else (that is, the
        		//"event handled" is true)
  }
    
// Continued on the next page -->
\end{verbatim}

\begin{verbatim}
  // Called when the user draws with the mouse button down
  // Draws a line between the points (last_x, last_y) and (x, y)
    
  public boolean mouseDrag(Event event, int x, int y)
  {
    Graphics g = this.getGraphics(); //Necessary to draw 
    g.setColor(current_color);	 //Line will be current_color
    g.drawLine(last_x, last_y, x, y);//Draws the line
    last_x = x;	//Saves (x,y) in last (last_x, last_y), so that
    last_y = y;	//the next line drawn will start from these points
    return true;	//and don't do anything else ("event handled").
  }

  // Called when the user clicks the Clear button, and 
  // clears the graphics window.
    
  public boolean action(Event event, Object arg) {
    //If Clear button was clicked, clear graphics window
    if (event.target == clear_button) {
      Graphics g = this.getGraphics();
      Rectangle r = this.bounds();
      g.setColor(this.getBackground());
      g.fillRect(r.x, r.y, r.width, r.height);
      return true;
    }
    
//ADDED:
else if (event.target == black_button)
{
  current_color = Color.black;
  return true;
}
else if (event.target == purple_button)
{
  current_color = Color.purple;
  return true;
}
    
    // Otherwise, let the superclass handle it.
    else return super.action(event, arg);
  }
}
\end{verbatim}

                        \item Add square root, {\tt sqrt}, 
			    to the {\tt expr} package
			    below.  {\tt sqrt} returns
			    the square root of its argument.
\begin{verbatim}
package expr;
import ccj.*;

public class Expression
{ public Expression(String expr)
  { parser = new Parser(expr);
    opstack = new java.util.Stack();
    numstack = new java.util.Stack();
  }
  public double eval()
  { while (true)
    { int type = parser.nextTokenType();
      String s = parser.nextToken();
      if (type == Parser.OPERATOR)
      { if (opstack.empty())
          opstack.push(s);
        else
        { String old = (String)opstack.pop();
          if (Parser.precedence(s) > Parser.precedence(old))
            opstack.push(old);
          else
            evalOperator(old);
          opstack.push(s);
        }
      }
      else if (type == Parser.LEFT_PAREN)
        opstack.push(s);
      else if (type == Parser.RIGHT_PAREN)
      { boolean more = true;
          while (more)
          { if (opstack.empty())
              throw new IllegalArgumentException("Too many )");
            String old = (String)opstack.pop();
            if (old.equals("(")) more = false;
            else evalOperator(old);
          }
      }
      else if (type == Parser.END_OF_STRING)
      { while (!opstack.empty())
          { String old = (String)opstack.pop();
            if (old.equals("("))
              throw new IllegalArgumentException("Too many (");
            else evalOperator(old);
          }
          if (numstack.empty())
            throw new IllegalArgumentException("Syntax error");
          double x = Numeric.parseDouble((String)numstack.pop());
         
          if (!numstack.empty())
            throw new IllegalArgumentException("Syntax error");
          return x;
      }
      else if (type == Parser.NUMBER)
        numstack.push(s);
      else
        throw new IllegalArgumentException("Bad token");
    }
  }

  private void evalOperator(String op)
  { if (numstack.empty())
      throw new IllegalArgumentException("Syntax error");
    double y = Numeric.parseDouble((String)numstack.pop());
    if (numstack.empty())
      throw new IllegalArgumentException("Syntax error");
      
      
//ADDED:
if (op.equals("sqrt") 
{
  if (x < 0)
    throw new IllegalArgumentException("Negative square root");
  z = Math.sqrt(x);
  numstack.push("" + z);
  return;
}
    
    double x = Numeric.parseDouble((String)numstack.pop());

    double z;
    if (op.equals("^")) z = Math.pow(x, y);
    else if (op.equals("*")) z = x * y;
    else if (op.equals("/"))
    { if (y == 0)
        throw new IllegalArgumentException("Divide by 0");
      else z = x / y;
    }
    else if (op.equals("+")) z = x + y;
    else if (op.equals("-")) z = x - y;
    else
      throw new IllegalArgumentException("Bad token");
    numstack.push("" + z);
  }

  private java.util.Stack opstack;
  private java.util.Stack numstack;
  private Parser parser;
}
\end{verbatim}

\begin{verbatim}

class Parser
{ Parser(String toParse)
  { input = toParse;
    pos = 0;
  }

  public int nextTokenType()
  { skipWhiteSpace();
    if (pos >= input.length()) return END_OF_STRING;
    String ch = input.substring(pos, pos + 1);
    if (ch.equals("(")) return LEFT_PAREN;
    if (ch.equals(")")) return RIGHT_PAREN;
    if ("+-*/^".indexOf(ch) != -1) return OPERATOR;
    if ("1234567890".indexOf(ch) != -1) return NUMBER;
    
//ADDED:
if (ch.equals("sqrt")) return OPERATOR;
    
    return BAD_TOKEN;
  }

  public String nextToken()
  { int i = nextTokenType();
    if (i == END_OF_STRING || i == BAD_TOKEN) return "";
    if (i == NUMBER)
    { int end = pos + 1;
      boolean more = true;
      while (more && end < input.length())
        { String ch = input.substring(end, end + 1);
          if (ch.equals(".") || "1234567890".indexOf(ch) != -1)
             end++;
          else
             more = false;
        }
      String r = input.substring(pos, end);
      pos = end;
      return r;
    }
    else
    { 

//ADDED:
if ( input.substring(pos,pos+4).equals("sqrt") )
{
  String r = input.substring(pos, pos + 4);
  pos = pos + 4;
  return r;
}

      String r = input.substring(pos, pos + 1);
      pos++;
      return r;
    }
  }
\end{verbatim}

\begin{verbatim}
   
  public static int precedence(String operator)
  { if (operator.equals("+") || operator.equals("-")) 
      return 1;
    if (operator.equals("*") || operator.equals("/")) 
      return 2;
    if (operator.equals("^")) return 3;
    
//ADDED:
if (operator.equals("sqrt") return 4;
    
    return 0; 
  }

  public static final int OPERATOR = 1;
  public static final int NUMBER = 2;
  public static final int LEFT_PAREN = 3;
  public static final int RIGHT_PAREN = 4;
  public static final int END_OF_STRING = 5;
  public static final int BAD_TOKEN = 6;

  private void skipWhiteSpace()
  { while (pos < input.length() 
      && input.substring(pos, pos + 1).equals(" "))
        pos++;
  }

  private String input;
  private int pos;
}
\end{verbatim}
                \end{enumerate}
            

%10
        \item  Design a class {\tt GradeBook}.  The {\tt GradeBook} must 
		keep track of student names (first, middle and last),
		student ID numbers (9 digits), and for each student
		up to 10 quiz scores, up to 30 program scores, and
		4 exam scores.	Also keep the current grade where
		quizzes are 20\%, programs are 40\% and exams are
		40\% of the total grade.

		Your design needs to include the definitions of
		the data for the class, the javadoc comments and
		definitions for the methods including constructors.
		{\bf Do NOT write the code for any of the methods!}
		
\begin{verbatim}

The definitions for the data:

  private String first;
  private String middle;
  private Strirng last;
  private int[] quizes;
  private int[] programs;
  private int[] exams;

A constructor with javadoc comment:
/**
 * The default constructor which allocates space for 10 quizzes,
 * 30 programs, and 4 exams.  Sets all strings to null and all 
 * numbers to 0
 **/
public GradeBook()

A method with javadoc comment:

/**
 * Calculates and returns the grade using the following 
 * formula: .2*(quiz total) + .4*(program total) + .4*(exam total).
 * Each total is calculated by summing up the values in the array
 * and dividing by the length. 
 *
 * @return the overall grade (which ranges from 0 to 100)
 **/
public double CalculateGrade()


\end{verbatim}

\end{enumerate}
\end{document}

