\documentstyle[11pt]{article}
\parskip 1em
\topmargin 1pt \headheight 1pt \headsep 2pt \footskip 1pt 
\oddsidemargin -10pt \evensidemargin -10pt \parindent 0pt
\setlength\textheight{700pt}
\setlength\textwidth{500pt}

\begin{document}

\input{epsf}

{\bf Computer Science 125-002 \hfill First Examination \hfill Fall 1998}

1. Assume that {\tt Foobar} is a class.  Draw a diagram which
illustrates the various relationships after executing the following
statements: [10 points]

\hspace*{.5in}{\tt Foobar a;}\\
\hspace*{.5in}{\tt Foobar b = new Foobar();}\\
\hspace*{.5in}{\tt Foobar a = b;}\\

2. For each of the following algebraic expressions, give an 
equivalent expression using correct Java syntax (your 
calculations should use {\tt double} quantities -- but
you can use any variable without declaration).  [7 points each]



(a) $\displaystyle{\frac{1}{a + b}}$

(b) $\displaystyle{b^2 - 4ac}$

(c) $\displaystyle{\sqrt{\frac{1}{1 + \frac{1}{x}}}}$


3. Write a complete stand-alone (non applet) Java program which
prompts a user for the weight of a first-class letter in ounces, computes
and displays the postage in cents for the letter according to the
following rules:

\begin{itemize}
\item The base charge for all letters is 33 cents.
\item For a letter weighing more than one ounce,
the added charge is 8 cents for each half ounce (or
fraction of) beyond 1 ounce.  For example, a calculation
for a letter of 
1.6 ounces can be computed using\\

\hspace*{.5in}33 + ceiling(2*(1.6-1.0))*8 cents or 49 cents.

The ceiling function ({\tt ceil}) is
part of the Math class (see page 63).
\end{itemize}

Use an {\tt int} function {\tt Charges} for calculating
the postage charges using a single {\tt double parameter}
which carries the weight of the letter.  Do all input
and output within the {\tt main} function. [12 points]

4. Consider the following recursive function (assume that some
suitable enclosing class definition exists). [12 points]

\begin{verbatim}
    public void funny(int m, int n)
    {
        if ( n > 0)
            return funny(n, m % n);
        else
            return m;
    }
\end{verbatim}

Suppose that the above function is called from another class
method as follows:

\hspace*{.5in}{\tt System.out.println(funny(15, 12));}

Show the chain of calls for the above and indicate the
value which is printed.



5. Explain the differences in usage between a static class
method and an instance method.  [8 points]


6. Indicate the output which is produced in the following
program: [10 points]

\begin{verbatim}
    public class Example
    {
        public static void main(String[] args)
        {
            int Value = 4;
            String s = new String("Hello");

            mystery(Value, s);
            System.out.println(Value + " " + s);
        }

        public static void mystery(int n, String foo)
        {
            n = 8;
            foo = "Bye";
        }
    }
\end{verbatim}



Explain your responses.



7. Write an {\tt if} or nested {\tt if} equivalent to
each of the following flow diagrams: [9 points each]

%\epsfysize=350pt\epsfbox{ifs.eps}

8. Simplify the following boolean expression as much as
possible: [9 points]

\hspace*{.5in}{\tt (!( !(x>1)  || (y<3) ))}
\end{document}



