\documentstyle[12pt]{article}
\pagestyle{empty}
\textwidth=6 true in
\textheight=9 true in
\topmargin = 10pt
\oddsidemargin = 0.0 true in
\evensidemargin = 0.0 true in
\newcommand{\ul}{\underline}
\newcommand{\spa}{\hspace{.25in}}
\begin{document}

{\Large
\begin{center}
\mbox{ }
    Exam 1 \\
    Computer Science 125 \\ 
    Boise State University\\ 
    Thursday, 24 September 1998
\end{center}
}

\begin{enumerate}

%1
   \item TRUE or FALSE:
	\begin{enumerate}
	    \item \underline{\hspace{.25in}} The type {\tt double}
		is an example of an object.
	    \item \underline{\hspace{.25in}} The logical expression
		$!(A\&\&B)$ is equivalent to $!A\&\&!B$.
	    \item \underline{\hspace{.25in}} The logical expression
		$A\&\&B$ evaluates $B$ first, then $A$.
	    \item \underline{\hspace{.25in}} A function has exactly
		one return value.
	    \item \underline{\hspace{.25in}} A function has at most
		one return value.
	    \item \underline{\hspace{.25in}} A function with no
		parameters must have a {\tt void} return type.
	    \item \underline{\hspace{.25in}} Object variables hold
		references, not values.
	    \item \underline{\hspace{.25in}} Every object belongs
		to a class.
	    \item \underline{\hspace{.25in}} If an action is 
		legal, than it's ethical.
	    \item \underline{\hspace{.25in}} Those caught committing
		software crime are most often low-level employees.
	\end{enumerate}

%2
    \item Give the Java code to construct the following objects:
	\begin{enumerate}
	    \item The current time:
	    \item Your next birthday:
	    \item The president as an employee (use \$200,000 as
		the salary):
	    \item A circle with center $(-1,-1)$ and radius $3$:
	    \item A line running from $(-10,-10)$ to $(10,10)$:
	\end{enumerate}

%3
    \item What is the output? 
	\begin{enumerate}
	    \item 
\begin{verbatim}
String s = "exam";
boolean okay = false;

if ( s.substring(1,2).equals("e") )
   okay = true;

if ( okay )
   System.out.println("Okay!");
else
   System.out.println("Not Okay!");
\end{verbatim}

	    \item 
\begin{verbatim}
int year = 1900;
boolean leap = (year % 4 == 0) && !(year % 100 == 0);
leap = leap || (year % 400 == 0);

if ( leap )
   System.out.println("Leap!");
else
   System.out.println("Not Leap!");
\end{verbatim}
    \end{enumerate}




%4
    \item What does this program produce as output?
\begin{verbatim}
import ccj.*;
public class Mystery
{ public static void main(String[] args)
    { int x = 12345;
       if ( myst(x) % 3 == 0 )
          System.out.println("Divisible!");
       else
          System.out.println("Indivisible!");
    }
    public static int myst(int x)
    { if ( x < 10 )
          return x;
       return ( (x%10) + myst(x/10) );
    }
}
\end{verbatim}


%5
    \item What does this program produce as output?
\begin{verbatim}
import ccj.*;
public class MysteryApplet extends GraphicsApplet
{ public void run()
    {  Point p1 = new Point(-5,-5);
       Circle c1 = new Circle(p1, 4);
       c1.draw();
       p1.move(7,7);
       p1.draw(); 
       Line l1 = new Line(p1, new Point(0,0));
       l1.draw();
    }
}
\end{verbatim}

%6
    \item Write a function that takes 2 points as input and
	returns the distance between the points.
	{\it (Recall that the distance between $(x_1,y_1)$
	and $(x_2,y_2)$ is $\sqrt{{(x_1-x_2)}^2 + {(y_1 - y_2)}^2}$.)}


%7
    \item Write a recursive function that takes a string and
	returns true if the string is a palindrome (the same
	written backwards as forwards).

%8
    \item A friend gives you a nice Java package that helps
	you finish your project for your new boss.  You
	later find out that this nice Java package is 
	copyrighted by the competitor.  No one is likely
	to find out that you are using this package (it
	has been compiled into bytecode).  However, if they
	do, you will be fired and your company sued.
	What should you do? 

	\begin{enumerate}
	    \item Who has something at stake in this dilemma?
	    \item What should you do?  Briefly explain why:  
	    \item Is what you did legal?  Briefly explain why:  
	    \item What added knowledge would help you make
		this decision?  Or cause you to change your
		decision?  Briefly explain why:
	\end{enumerate}
%9
    \item Given the following function definitions:
\begin{verbatim}
public static double hare(double x) { return 2 * hatter(x/2); }
public static double hatter(double x) { return x * alice(x); }
public static double alice (double x) { return mouse(x) - 1 ; }
public static double mouse(double x) { return x*x - x; }
\end{verbatim}

    \begin{enumerate}
	\item {\tt alice(5) = } \underline{\hspace{2in}}
	\item {\tt hare(4) = } \underline{\hspace{2in}}
    \end{enumerate}
	
%10
    \item Write a recursive function that takes an integer
	and breaks it into a sequence of individual digits.
	For example, 16384 would be displayed as
\begin{verbatim}
1 6 3 8 4
\end{verbatim}
	{\it (Hint: Use recursion.)}

\end{enumerate}
\end{document}


