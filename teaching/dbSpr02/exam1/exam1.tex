\documentstyle[12pt]{article}
\pagestyle{empty}
\topmargin= -25pt
\textwidth=6 true in
\textheight=9.5 true in
\oddsidemargin = 0.0 true in
\evensidemargin = 0.0 true in
\newcommand{\ul}{\underline}
\newcommand{\spa}{\hspace{.25in}}
\begin{document}

{\large
\begin{center}
%\mbox{ }
%\vspace{.1in} \\
    Exam I\\
    Computer Science 420 \\
    Dr.~St.~John\\ 
    Lehman College\\
    City University of New York\\ 
    11 October 2001
\end{center}
}

{\large
\begin{center}
\begin{tabular}{ll}
\mbox{ }
\vspace{.1in} \\
NAME (Printed) & \ul{\hspace{3in}}\\ 
NAME (Signed) & \ul{\hspace{3in}} \\
E-mail & \ul{\hspace{3in}}\\
\end{tabular}
\end{center}


{\small
{\bf
\begin{center}
Exam Rules
\end{center}
\begin{itemize}
	\item Show all your work.  Your grade will be based on the work shown.          \item The exam is closed book and closed notes. 
        \item When taking the exam, you may have with you pens or pencils,
		and an 8 1/2" x 11" 
		piece of paper filled with notes, programs, etc. 
	\item You may not use a computer or calculator. 
	\item All books and bags must be left at the front of the classroom 
		during this exam. 
	\item {\bf Do not open this exams until instructed to do so. }
\end{itemize}
}

{\large
\begin{center}
\mbox{ }
\vspace{.25in} \\
\begin{tabular}{|c|l|}
\hline \hline
\hspace{.05in} Question 1 \hspace{.05in} & \hspace{.5in} \mbox{ }  \\ \hline
\hspace{.05in} Question 2 \hspace{.05in} & \hspace{.5in} \mbox{ } \\ \hline
\hspace{.05in} Question 3 \hspace{.05in} & \hspace{.5in} \mbox{ } \\ \hline
\hspace{.05in} Question 4 \hspace{.05in} & \hspace{.5in} \mbox{ } \\ \hline
\hspace{.05in} Question 5 \hspace{.05in} & \hspace{.5in} \mbox{ }  \\ \hline 
\hspace{.05in} Question 6 \hspace{.05in} & \hspace{.5in} \mbox{ }  \\ \hline
\hspace{.05in} Question 7 \hspace{.05in} & \hspace{.5in} \mbox{ }  \\ \hline
\hspace{.05in} Question 8 \hspace{.05in} & \hspace{.5in} \mbox{ } \\ \hline
\hspace{.05in} Question 9 \hspace{.05in} & \hspace{.5in} \mbox{ } \\ \hline
\hspace{.05in} Question 10\hspace{.05in} & \hspace{.5in} \mbox{ } \\ \hline

\hspace{.05in} TOTAL \hspace{.05in} & \hspace{1in} \mbox{ }  \\ \hline \hline
\end{tabular}
\end{center}
}

\newpage



\begin{enumerate}

% 1 T or F
    \item True or False: 
    \begin{enumerate}
        \item \underline{\hspace{.25in}} All database management systems
		are relational.
        \item \underline{\hspace{.25in}} Every set is a bag.
        \item \underline{\hspace{.25in}} Relationships cannot have attributes
		in E/R diagrams. 
        \item \underline{\hspace{.25in}} Relationships in E/R diagrams 
		connect one or more entity sets.
        \item \underline{\hspace{.25in}} In ODL, every relationship must
		have an inverse.
        \item \underline{\hspace{.25in}} Every key is a superkey.
        \item \underline{\hspace{.25in}} Every superkey is a key.
        \item \underline{\hspace{.25in}} Attributes that are keys cannot 
		appear in functional dependencies.
        \item \underline{\hspace{.25in}} If $X$ is a key for the relation			R, then there is functional dependency for R with 
		$X \rightarrow \mbox{all attributes}$.
        \item \underline{\hspace{.25in}} Every functional dependency is
		a multivalued dependency.
    \end{enumerate}

%2 Short answer
\item Answer in two sentences or less the following:
\begin{enumerate}
    \item Why are there no ``weak classes'' in ODL (but there are weak
	entity sets in E/R diagrams)?
	\vspace{.75in}
    \item What is an anomaly?  Give an example.
	\vspace{.75in}
\end{enumerate}

%3 Given R and functional dependencies, give keys and superkeys
\item Consider the relation $R(A,B,C,D,E)$ with the function
dependencies: 
$$
	A\rightarrow B,
	B\rightarrow C,
	C \rightarrow A,
	D \rightarrow E, \mbox{ and }
	E\rightarrow D
$$
\begin{enumerate}
    \item What are the keys of $R$?  
	\vspace{1in}
    \item How many superkeys are there of $R$?  Justify your answer.
	\vspace{1in}
\end{enumerate}

\newpage
%4 From an E/R diagram, give keys specified and translate into
% database schema
\item Draw an E/R diagram for the following situations. Indicate any
keys, weak entity sets, or subclasses.
\begin{enumerate}
    \item Entity sets {\em Courses} and {\em Departments}.  A course
	is given by a unique department, but its only attribute is its
	number.  Different departments can offer courses with the same
	number.  Each department has a unique name.
	\vspace{3in}
    \item Entity sets {\em Computers}, {\em Laptops}, and {\em Owners}.
	A computer has a manufacturer, CPU speed, and a unique name.
	Laptops, have all the properties of computers, as well as their
	weight and battery life.  Owners are identified by their names.
	Every computer has at most one owner, but owners can have 
	several computers.
\end{enumerate}

\newpage
%5 From an ODL definition, give keys specified and translate into
% database schema
\item Give an ODL design for a database recording information about teams,
players, and their fans, including:
\begin{enumerate}
    \item For each team, its name, its players, its team captain (one its
	players), and the colors of its uniform.
    \item For each player, his/her name.
    \item For each fan, his/her name, favorite teams, favorite players,
	and favorite color.
\end{enumerate}

\newpage
%6 Many-one, one-one, many-many
\item For each of the following types of relationships, give an
	example and draws its E/R diagram:
    \begin{enumerate}
	\item one-one:
	\vspace{1.2in}
	\item many-one:
	\vspace{1.2in}
	\item many-many:
	\vspace{1.2in}
    \end{enumerate}
%8 Functional Dependencies and Multivalued Dependencies
\item 
  \begin{enumerate}
    \item Consider a relation $R(A,B)$ with two tuples: $R=\{(4,1),(4,2)\}$.
    \begin{enumerate}
	\item Does $A\rightarrow B$ hold for this instance of $R$?\\
	    \smallskip
	    Circle one: YES \ \ NO
	    \bigskip
	\item Does $A\rightarrow\!\hspace{-0.5em}\rightarrow B$ hold for 
	    this instance of $R$?\\
	    \smallskip
	    Circle one: YES \ \ NO
	    \bigskip
	    \bigskip
    \end{enumerate}
    \item Now consider a relation $R(A,B,C)$ with two tuples:
    $R = \{(3,2,1), (4,2,6)\}$.
    \begin{enumerate}
	\item Does $A\rightarrow B$ hold for this instance of $R$?\\
	    \smallskip
	    Circle one: YES \ \ NO
	    \bigskip
	\item Does $B\rightarrow C$ hold for this instance of $R$?\\
	    \smallskip
	    Circle one: YES \ \ NO
	    \bigskip
	\item Does $A\rightarrow\!\hspace{-0.5em}\rightarrow C$ 
	    hold for this instance of $R$?\\
	    \smallskip
	    Circle one: YES \ \ NO
	    \bigskip
	    \bigskip
	    \bigskip
    \end{enumerate}
    \end{enumerate}

\newpage
%6 odl --> relational database schema
\item Convert the following ODL description of a schema to a relational
database schema.  Remember that {\tt Course} objects have an ``object
identity,'' and you may invent an attribute representing this OID, 
e.g. {\tt CourseID}.  

\begin{verbatim}
interface Course {
    attribute integer number;
    attribute string room;
    relationship Dept deptOf inverse Dept::coursesOf;
};

interface LabCourse : Course {
    attribute integer computerAlloc;
};

interface Dept (key name) {
    attribute string name;
    attribute string chair;
    relationship Set<Course> coursesOf
         inverse Dept::deptOf;
};
\end{verbatim}

\newpage


\newpage
%8 Compare/Contrast Normal Forms
\item 
    \begin{enumerate} 
	\item Define Boyce-Codd Normal Form (BCNF):
	    \vspace{1in}
	\item Define Third Normal Form (3NF):
	    \vspace{1in}
	\item Define Fourth Normal Form (4NF):
	    \vspace{1in}
        \item Is every relation in Third Normal Form also in Boyce Codd
	    Normal Form?  If yes, explain why.  If no, give an example
	    that shows why this is not true.
	    \vspace{1.75in}
        \item Is every relation in Fourth Normal Form also in Boyce Codd
	    Normal Form?  If yes, explain why.  If no, give an example
	    that shows why this is not true.
	    \vspace{1.5in}
    \end{enumerate} 

\newpage

%10 BCNF and 4NF
\item Given the relation schema $R(A,B,C,D)$ with the functional dependencies
$$
\begin{array}{c}
	AB \rightarrow C\\
	BC \rightarrow D\\
	CD \rightarrow A\\
	AD \rightarrow B\\
\end{array}
$$
\begin{enumerate}
    \item Indicate all the Boyce Codd Normal Form violations.  
	Do not forget to consider
	dependencies that are not in the given set, but follow from them.
	However, it is not necessary to give violations that have more than
	one attribute on the right side.
	\vspace{2.5in}
    \item Decompose the relations, as necessary, into a collection of 
	relations that are in Boyce Codd Normal Form.
	\vspace{2in}
\end{enumerate}


\end{enumerate}
\end{document}


