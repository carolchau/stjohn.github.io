\documentclass[12pt]{article}
\pagestyle{empty}
\topmargin= -25pt
\textwidth=6 true in
\textheight=9.5 true in
\oddsidemargin = 0.0 true in
\evensidemargin = 0.0 true in
\newcommand{\ul}{\underline}
\newcommand{\spa}{\hspace{.25in}}
\begin{document}

{\large
\begin{center}
%\mbox{ }
%\vspace{.1in} \\
    Answer Key for Exam I\\
    Computer Programming 420 \\
    Dr.~St.~John\\ 
    Lehman College\\
    City University of New York\\ 
    11 October 2001
\end{center}
}


\begin{enumerate}

% 1 T or F
    \item True or False: 
    \begin{enumerate}
        \item \underline{F} All database management systems
		are relational.
        \item \underline{F} Every set is a bag.
        \item \underline{F} Relationships cannot have attributes
		in E/R diagrams. 
        \item \underline{T} Relationships in E/R diagrams 
		connect one or more entity sets.
        \item \underline{T} In ODL, every relationship must
		have an inverse.
        \item \underline{T} Every key is a superkey.
        \item \underline{F} Every superkey is a key.
        \item \underline{F} Attributes that are keys cannot 
		appear in functional dependencies.
        \item \underline{T} If $X$ is a key for the relation			
		R, then there is functional dependency for R with 
		$X \rightarrow \mbox{all attributes}$.
        \item \underline{T} Every functional dependency is
		a multivalued dependency.
    \end{enumerate}

%2 Short answer
\item Answer in two sentences or less the following:
\begin{enumerate}
    \item Why are there no ``weak classes'' in ODL (but there are weak
	entity sets in E/R diagrams)?

	{\tt Because every object has an unique object id (OID), and
	so, has a key.  Weak entity sets occur when a secondary 
	entity set provides part of the key.}

    \item What is an anomaly?  Give an example.

	{\tt An anomaly is a problem usually caused when too much 
	information is crammed into one relation.  For example, a 
	redundacy anomaly can occur when information is stored in
	duplicate times in the table.}
\end{enumerate}

%3 Given R and functional dependencies, give keys and superkeys
\item Consider the relation $R(A,B,C,D,E)$ with the function
dependencies: 
$$
	A\rightarrow B,
	B\rightarrow C,
	C \rightarrow A,
	D \rightarrow E, \mbox{ and }
	E\rightarrow D
$$
\begin{enumerate}
    \item What are the keys of $R$?  

	{\tt The keys are: $AD$, $AE$, $BD$, $BE$, $CD$, and $CE$.}

    \item How many superkeys are there of $R$?  Justify your answer.

	{\tt Recall that a superkey is key or a superset of a key.
	So, every superkey must consist of at least one of $\{A,B,C\}$
	and $\{D,E\}$ (since every key has at least one element of
	each).  This gives that the number of superkeys is the number
	of ways to choose 1, 2, or 3 attributes from $\{A,B,C\}$ times
	the number of ways to choose 1 or 2 attributes from $\{D,E\}$, or:
	$$
	\begin{array}{c}
		\left(
		\left( \begin{array}{c} 3 \\ 1 \end{array} \right)
		+ \left( \begin{array}{c} 3 \\ 2 \end{array} \right)
		+ \left( \begin{array}{c} 3 \\ 3 \end{array} \right)
		\right)
		\cdot
		\left(
		\left( \begin{array}{c} 2 \\ 1 \end{array} \right)
		\left( \begin{array}{c} 2 \\ 2 \end{array} \right)
		\right)
		
		= (1+3+3)\cdot(1+2)
		= 21
	\end{array}
	$$
	
	Another way to do this question is to list out all possible
	superkeys and then count them:

	For example, the superkeys of size 2 and 3 can be formed by:
	$$
	\begin{array}{c}
	\mbox{the keys: } AD, AE, BD, BE, CD, CE\\
	\mbox{adding A: } ABD, ABE, ACD, ACE\\
	\mbox{adding B: } ABD, ABE, BCD, BCE\\
	\mbox{adding C: } ACD, ACE, BCD, BCE\\
	\mbox{adding D: } ADE, BDE, CDE\\
	\mbox{adding E: } ADE, BDE, CDE\\
	\end{array}
	$$
	Removing the duplicates leaves:
	$$
	\begin{array}{c}
	AD, AE, BD, BE, CD, CE\\
	ABD, ABE, ACD, ACE\\
	BCD, BCE\\
	ADE, BDE, CDE\\
	\end{array}
	$$
	or 9 superkeys of size 2 and 6 superkeys of size 3.

	To get the keys of size 4:
	$$
	\begin{array}{c}
	\mbox{of size 3: } ABD, ABE, ACD, ACE, BCD, BCE, ADE, BDE, CDE\\
	\mbox{adding A: } ABCD, ABCE, ABDE, ACDE\\
	\mbox{adding B: } ABCD, ABCE, ABDE, BCDE\\
	\mbox{adding C: } ABCD, ABCE, ACDE, BCDE\\
	\mbox{adding D: } ABDE, ACDE, BCDE\\
	\mbox{adding E: } ABDE, ACDE, BCDE\\
	\end{array}
	$$
	and remove duplicates to yield
	$$
	\begin{array}{c}
	ABCD, ABCE, ABDE, ACDE\\
	BCDE\\
	\end{array}
	$$
	and remove duplicates to yield 5 keys of size 4.

	Since there's only one key with all 5 attributes, this gives
	$6+9+5+1 = 21$ possible superkeys.


	}

\end{enumerate}

\newpage
%4 From an E/R diagram, give keys specified and translate into
% database schema
\item Draw an E/R diagram for the following situations. Indicate any
keys, weak entity sets, or subclasses.
\begin{enumerate}
    \item Entity sets {\em Courses} and {\em Departments}.  A course
	is given by a unique department, but its only attribute is its
	number.  Different departments can offer courses with the same
	number.  Each department has a unique name.

    \item Entity sets {\em Computers}, {\em Laptops}, and {\em Owners}.
	A computer has a manufacturer, CPU speed, and a unique name.
	Laptops, have all the properties of computers, as well as their
	weight and battery life.  Owners are identified by their names.
	Every computer has at most one owner, but owners can have 
	several computers.
\end{enumerate}

%5 From an ODL definition, give keys specified and translate into
% database schema
\item Give an ODL design for a database recording information about teams,
players, and their fans, including:
\begin{enumerate}
    \item For each team, its name, its players, its team captain (one its
	players), and the colors of its uniform.
    \item For each player, his/her name.
    \item For each fan, his/her name, favorite teams, favorite players,
	and favorite color.
\end{enumerate}

\newpage
%6 Many-one, one-one, many-many
\item For each of the following types of relationships, give an
	example and draws its E/R diagram:
    \begin{enumerate}
	\item one-one:
	\item many-one:
	\item many-many:
    \end{enumerate}
%8 Functional Dependencies and Multivalued Dependencies
\item 
  \begin{enumerate}
    \item Consider a relation $R(A,B)$ with two tuples: $R=\{(4,1),(4,2)\}$.
    \begin{enumerate}
	\item Does $A\rightarrow B$ hold for this instance of $R$?\\
	    \smallskip
	    Circle one: YES \ \ \underline{NO}
	    \bigskip
	\item Does $A\rightarrow\!\hspace{-0.5em}\rightarrow B$ hold for 
	    this instance of $R$?\\
	    \smallskip
	    Circle one: \underline{YES} \ \ NO
	    \bigskip
	    \bigskip
    \end{enumerate}
    \item Now consider a relation $R(A,B,C)$ with two tuples:
    $R = \{(3,2,1), (4,2,6)\}$.
    \begin{enumerate}
	\item Does $A\rightarrow B$ hold for this instance of $R$?\\
	    \smallskip
	    Circle one: \underline{YES} \ \ NO
	    \bigskip
	\item Does $B\rightarrow C$ hold for this instance of $R$?\\
	    \smallskip
	    Circle one: YES \ \ \underline{NO}
	    \bigskip
	\item Does $A\rightarrow\!\hspace{-0.5em}\rightarrow C$ 
	    hold for this instance of $R$?\\
	    \smallskip
	    Circle one: YES \ \ \underline{NO}
	    \bigskip
    \end{enumerate}
   \end{enumerate}

\newpage
%6 odl --> relational database schema
\item Convert the following ODL description of a schema to a relational
database schema.  Remember that {\tt Course} objects have an ``object
identity,'' and you may invent an attribute representing this OID, 
e.g. {\tt CourseID}.  

\begin{verbatim}
interface Course {
    attribute integer number;
    attribute string room;
    relationship Dept deptOf inverse Dept::coursesOf;
};

interface LabCourse : Course {
    attribute integer computerAlloc;
};

interface Dept (key name) {
    attribute string name;
    attribute string chair;
    relationship Set<Course> coursesOf
         inverse Dept::deptOf;
};
\end{verbatim}

\newpage


\newpage
%8 Compare/Contrast Normal Forms
\item 
    \begin{enumerate} 
	\item Define Boyce-Codd Normal Form (BCNF):
	    \vspace{1in}
	\item Define Third Normal Form (3NF):
	    \vspace{1in}
	\item Define Fourth Normal Form (4NF):
	    \vspace{1in}
        \item Is every relation in Third Normal Form also in Boyce Codd
	    Normal Form?  If yes, explain why.  If no, give an example
	    that shows why this is not true.
	    \vspace{1.75in}
        \item Is every relation in Fourth Normal Form also in Boyce Codd
	    Normal Form?  If yes, explain why.  If no, give an example
	    that shows why this is not true.
	    \vspace{1.5in}
    \end{enumerate} 

\newpage

%10 BCNF and 4NF
\item Given the relation schema $R(A,B,C,D)$ with the functional dependencies
$$
\begin{array}{c}
	AB \rightarrow C\\
	BC \rightarrow D\\
	CD \rightarrow A\\
	AD \rightarrow B\\
\end{array}
$$
\begin{enumerate}
    \item Indicate all the Boyce Codd Normal Form violations.  
	Do not forget to consider
	dependencies that are not in the given set, but follow from them.
	However, it is not necessary to give violations that have more than
	one attribute on the right side.
	\vspace{2.5in}
    \item Decompose the relations, as necessary, into a collection of 
	relations that are in Boyce Codd Normal Form.
	\vspace{2in}
\end{enumerate}


\end{enumerate}
\end{document}


