\documentstyle[12pt]{article}
\pagestyle{empty}
\textwidth=6 true in
\textheight=9 true in
\topmargin = 10pt
\oddsidemargin = 0.0 true in
\evensidemargin = 0.0 true in
\newcommand{\ul}{\underline}
\newcommand{\spa}{\hspace{.25in}}
\begin{document}

{\Large
\begin{center}
\mbox{ }
    Exam 2 \\
    Computer Science 125 \\ 
    Dr.~St.~John\\
    Boise State University\\ 
    Monday, 9 November 1998
\end{center}
}

\begin{enumerate}

%1
   \item TRUE or FALSE:
	\begin{enumerate}
	    \item \underline{\hspace{.25in}} A constructor is
		used to initialize objects when they are created.

	    \item \underline{\hspace{.25in}} A class may have
		at most one method with the same name.

	    \item \underline{\hspace{.25in}} When creating an
		array, you do not need to specify the number
		of elements you want the array to hold.

	    \item \underline{\hspace{.25in}} Array subscripts must
		be integers.

	    \item \underline{\hspace{.25in}} Arrays cannot contain
		strings as subscripts.

	    \item \underline{\hspace{.25in}} If a program has passed
	    all tests in a test suite, it has no more bugs.

	    \item \underline{\hspace{.25in}} Functions declared as
		{\tt static} operate on objects and are called
		instance methods.

	    \item \underline{\hspace{.25in}} Every Java program can
		be rewritten to avoid {\tt import} statements.

	    \item \underline{\hspace{.25in}} You may no more than
		one {\tt public} class per file.

	    \item \underline{\hspace{.25in}} Stacks have 
		``first in, first out'' or FIFO access.


	\end{enumerate}

%2
	\item What are the values of {\tt s} and {\tt n} after
	    the following loops:
	    \begin{enumerate}
		\item 
\begin{verbatim}
int s = 1;
int n = 1;
while (s < 10) 
{ 
    s = s + n; 
    n++; 
}
\end{verbatim}
		\item
\begin{verbatim}
int s = 1;
int n = 1;
do { 
    s = s + n; 
    n++; 
} while ( s < 2*n );
\end{verbatim}
	    \end{enumerate}

%3
    \item Consider the following class.  
	\begin{enumerate}
	    \item Mark the constructors
		with a ``C'', the accessor functions with a ``A'',
		and the mutator functions with a ``M'':
\begin{verbatim}

class A
{
    public A() { n = 0; }

    public A(int a) { n = a; }

    public void f() { n++; }

    public void g() { f(); n = 2 * n; f(); }

    public void h() { return n; }

    public void k() { System.out.println(n); }

    private int n;
}

\end{verbatim}

	    \item  Add a method {\tt c()} to the class above that
		divides {\tt n} by 2 if {\tt n} is even and 
		replaces it with {\tt 3 * n + 1} if {\tt n} is odd.

	\end{enumerate}

%4
    \item Write a function:
\begin{verbatim}
public boolean isPrime(int n)
\end{verbatim}
	that returns true if {\tt n} is a prime number and 
	false otherwise.
	

%5
    \item For each of the following sets of values, write code that
	fills an array {\tt v} with the values:
	\begin{enumerate}
	    \item {\tt 10 20 30 40 50 60 70 80 90 100}
	    \item {\tt 3 5 3 5 3 5 3 5 3 5 }
	\end{enumerate}

%6
    \item What is the output from the following programs:
	\begin{enumerate}
	    \item 
\begin{verbatim}
import ccj.*;
public class mystery1 extends GraphicsApplet
{  public void run()
    {   Point[] pp = new Point[6];
        for ( int i = 0 ; i < 6 ; i++ )
        {   double t = i * (2*Math.PI/6);
            pp[i] = new Point(5*Math.cos(t),5*Math.sin(t));
        }       
        for ( int i = 0 ; i < n ; i++ )
            new Line(pp[i], pp[(i+1)%6]).draw();
    }
}
\end{verbatim}

	    \item
\begin{verbatim}
import ccj.*;
public class mystery2 extends GraphicsApplet
{   public void run()
    {   Point[] pp = new Point[8];
        for ( int i = 0 ; i < 8 ; i++ )
        {   double t = i * (2*Math.PI/8);
            pp[i] = new Point(5*Math.cos(t),5*Math.sin(t));
        }       
        new Circle(new Point(0,0),5).draw();
        for ( int i = 0 ; i < 8 ; i++ )
            new Line(pp[i], new Point(0,0)).draw();
    }
}
\end{verbatim}
	\end{enumerate}

%7
    \item 
	\begin{enumerate}
	    \item What is the difference between a virus and a Trojan 
	    	horse?  Give examples.
	    \item What were Social Security Numbers (SSN) originally
	    	designed for?  Why are SSN used as ``passwords'' for
	    	consumer databases such as credit bureaus?  Why is 
	    	this a bad choice?
	    \item Data generated by a computer program is used to 
	    	design a bridge.  The bridge collapses during the 
		last week of construction killing two people.
	    	Who is responsible?  Why?
	\end{enumerate}


%9
    \item Modify the following program to allow the user to change 
	the colors that are drawn.   Add at least {\bf 2 choices of colors} and
    	buttons for each choice. For example, if you choose the colors 
	red and black, you should have buttons that when pressed,
    	change the color that is drawn to red or black, respectively. 
	Your applet should also have a clear button like OneColorDraw.
\begin{verbatim}

import java.applet.*;   
import java.awt.*;     

public class OneColorDraw extends Applet 
{
    private int last_x = 0;     
    private int last_y = 0;    
    private Button clear_button;
    private Color current_color = Color.black;  

    
    public void init() 
    {
        clear_button = new Button("Clear");
        clear_button.setForeground(Color.black);
        clear_button.setBackground(Color.lightGray);
        this.add(clear_button);           






    }

    public boolean mouseDrag(Event event, int x, int y)
    {
        Graphics g = this.getGraphics(); 
        g.setColor(current_color);      
        g.drawLine(last_x, last_y, x, y);
        last_x = x;     
        last_y = y;    
        return true;  
    }


    // (continued on the next page -->)
\end{verbatim}
\begin{verbatim}


    
    public boolean action(Event event, Object arg) 
    {

        if (event.target == clear_button) {
            Graphics g = this.getGraphics();
            Rectangle r = this.bounds();
            g.setColor(this.getBackground());
            g.fillRect(r.x, r.y, r.width, r.height);
            return true;
        }
        else return super.action(event, arg);

    }


}
\end{verbatim}

%10
    \item Fill in the missing function definitions in the {\tt PuzzlePiece}
	class:
\begin{verbatim}import ccj.*;
public class PuzzlePiece 
{   
    private Point ul;
    int label;
    /**
     * Initialize the label to 0 and the corner to (0,0).
     */
    public PuzzlePiece()
    {




    }
    /**
     * Initialize the label to l and the corner to (0,0).
     * @param l the label for the piece
     */
    public PuzzlePiece(int l)
    {




    }
    /**
     * Initialize the label to l and the corner to (x,y).
     * @param l the label for the piece
     * @param x the x-coordinate of the corner.
     * @param y the y-coordinate of the corner.
     */
    public PuzzlePiece(int l, double x, double y)
    {





    }

    // (continued on the next page -->)
\end{verbatim}
\begin{verbatim}


    /**
     * Move the piece by dx units in the x direction and dy units in the y 
     * direction.
     * @param dx change in x-coordinate
     * @param dy change in y-coordinate
     */
    public void move(int dx, int dy)
    {










    }

    /**
     * Draw the puzzle piece.  The label is drawn, surrounded by
     * a square whose upper left corner is the point stored in
     * PuzzlePiece.
     */
    public void draw()
    {
        Point ll = new Point(ul.getX(), ul.getY()+2);
        Point lr = new Point(ul.getX()+2, ul.getY()+2);
        Point ur = new Point(ul.getX()+2, ul.getY());
        new Line( ul, ll).draw();
        new Line( ul, ur).draw();
        new Line( ll, lr).draw();
        new Line( ur, lr).draw();
        Point mp = new Point(ul.getX()+1, ul.getY()+1);
        Message m = new Message(mp, "" + label);
        m.draw();
    }
}
\end{verbatim}

%10
    \item Write a {\bf complete} program that counts the
    	number of lines in a file.  For example,
\begin{verbatim}
$ java Count sample.txt
\end{verbatim}
	counts the number of lines in the file {\tt sample.txt}.
	

\end{enumerate}
\end{document}
