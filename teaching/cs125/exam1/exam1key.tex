\documentstyle[12pt]{article}
\pagestyle{empty}
\textwidth=6 true in
\textheight=9 true in
\topmargin = 10pt
\oddsidemargin = 0.0 true in
\evensidemargin = 0.0 true in
\newcommand{\ul}{\underline}
\newcommand{\spa}{\hspace{.25in}}
\begin{document}

{\Large
\begin{center}
    Key for Exam 1 \\
    Computer Science 125 \\
    Boise State University\\
    Friday, 19 February 1999
\end{center}
}

\begin{enumerate}

%1
   \item TRUE or FALSE:
        \begin{enumerate}
            \item \underline{F} The {\em Cuckoo's Egg} is fiction.
\smallskip
            \item \underline{T} The logical expression
                $!(A\&\&B)$ is equivalent to $!A||!B$.
\smallskip
            \item \underline{T} The logical expression
                $A\&\&B$ evaluates $B$ only if $A$ is {\tt false}.
\smallskip
            \item \underline{T} Cliff Stoll created the
            {\em Strategic Defense Initiative Network} as bait for the hacker.
\smallskip
            \item \underline{F} To execute an applet you type {\tt java} followed by the filename.
\smallskip
            \item \underline{F} David La Macchia was convicted of wire fraud.
\smallskip
            \item \underline{F} Objects should always be compared
            with ==.
\smallskip
            \item \underline{F} The compiler will locate all of the errors in every program.
\smallskip
            \item \underline{F} Most computer crimes are
            committed by professional computer programmers.
\smallskip
            \item \underline{T} It violates the BSU Math/CS
            Departmental computer policy to let your friend use your account
            to read their e--mail.
\end{enumerate}

%2
    \item Give the Java code to construct the following objects:
\begin{enumerate}
            \item The time spring break officially begins this semester\\
            (Hint: it starts March 19 at 5:00pm).
\begin{verbatim}
Time SpringBreak = new Time(1999,3,19,17,0,0);
\end{verbatim}

            \item An employee named Sam Smith whose salary is \$28,000.
\begin{verbatim}
Employee peon = new Employee("Sam Smith", 28000);
\end{verbatim}


            \item A line through $(0,0)$ which has slope $2$.
\begin{verbatim}
Line l = new Line(new Point(0,0),new Point(1,2));
\end{verbatim}


\end{enumerate}
%3
    \item What is the output for each of the following sections of code?
    There are no intentional syntax errors.
\begin{enumerate}
\item

\begin{verbatim}
String s = "the fun begins";
String t = "now";
String m = "later";

System.out.print(s + m+" or");
System.out.println(t + s);

\end{verbatim}

       {\bf Output:}
        \begin{tabular}{|c|}
            \hline
            the fun beginslater ornowthe fun begins\\
            \hline
        \end{tabular}

\item
\begin{verbatim}
String s = "123,457";
int n = s.length();
System.out.println(s.substring(n-4,n);
\end{verbatim}

        {\bf Output:}
        \begin{tabular}{|c|}
            \hline
            ,457\\
            \hline
        \end{tabular}

\end{enumerate}

    \item What is the output for the following sections of code?
    There are no intentional syntax errors.
\begin{verbatim}
int choice = 3;
boolean valid = true;
if (choice == 1)
{ System.out.println("You selected 1");
  valid = true;
}
else if (choice == 2)
{ System.out.println("You selected 2");
  valid = true;
}
else
{ System.out.println("Your choice is not valid");
  valid = false;
}

\end{verbatim}

        {\bf Output:}
        \begin{tabular}{|c|}
            \hline
            Your choice is not valid\\
           
            \hline
        \end{tabular}

\smallskip



%4
\item
    \begin{enumerate}\item What does this program produce as output?
\begin{verbatim}
import ccj.*;
public class Exam
{public static void main(String[] args)
  {  int A = 105;
     int B = A/60;
     System.out.println(""+(A+B));
  }
}
\end{verbatim}

\smallskip
\smallskip
        {\bf Output:}
        \begin{tabular}{|c|}
            \hline
            106\\
            \hline
        \end{tabular}

\smallskip
\smallskip

\item What does this program produce as output?
\begin{verbatim}
import ccj.*;
public class Example
{ public static void main(String[] args)
  { int x = 16;
    if ((x % 2 == 0) && (x % 3 == 0))
       System.out.println("The value of x is divisible by 6");
    else
       System.out.println("The value of x is not a multiple of 6");
  }
}
\end{verbatim}

        {\bf Output:}
        \begin{tabular}{|c|}
            \hline
            The value of x is not a multiple of 6\\
            \hline
        \end{tabular}
\end{enumerate}

%5

\item \begin{enumerate}
\item What does this program produce as output?

\begin{verbatim}
import ccj.*;
public class MysteryApplet1 extends GraphicsApplet
{public void run()
  {Point p1 = new Point(-3,4);
   Point p2 = new Point(3,4);
   Line line1 = new Line(p1,p2);
   Circle c = new Circle(new Point(0,4),3);
   c.draw();
   line1.draw();
  }
}
\end{verbatim}
        {\bf Output:}
%insert picture

    \item What does this program produce as output?
\begin{verbatim}
import ccj.*;
public class MysteryApplet extends GraphicsApplet
{ public void run()
    {  Point p1 = new Point(-5,-5);
       Point p2 = new Point(5,5);
       Line line1 = new Line(p1, p2);
       line1.draw();
       line1.move(3,3);
       line1.draw();
    }
}
\end{verbatim}
        {\bf Output:}
%insert picture here

\end{enumerate}

%6

%7
    \item Evaluate the following Java expressions.

 \begin{enumerate}
    \item $990 \% 100$ {\tt = 90}

    \item $4.4 * 2/4 + 6$ {\tt = 8.2}

    \item $8 \% 3 * 5 - 16/3 * 4$ {\tt = -10}

    \item $3.4 <= 8 - 9/2$ {\tt = true}

    \item $ x < 5 || x >= 5$ {\tt = true}

\end{enumerate}
%8
    \item Which of the ethical approaches discussed in class, Utilitarian,
    Rights, Fairness, Common--Good, or Virtue, comes closest to the
    way you make decisions?

\vspace{1cm}
    Answer the following question based on the approach that you selected.
    Justify your answer in terms of your chosen approach.

    In the {\em Cuckoo's Egg} Bob Morris (Sr.) was the head of
    the NSA and his son Robert T. Morris (Jr.) was responsible for
    writing a virus that brought the internet to a standstill. Was
    Bob Morris (Sr.) ethically bound to turn in his son?

%9
\item Below is an applet that will draw a cylinder. Modify the code
so that it will draw a cylinder whose radius and height are both doubled.
Cross out each of the statements that you are changing and write the new
statement beside the one that is changed.

\begin{verbatim}
import ccj.*;
public class Cylinder extends GraphicsApplet
{public void run()
 { Point origin = new Point(0,0);
   int radius = 3; //CROSS OUT THIS and add: int radius = 6;
   int height = 4; //CROSS OUT THIS and add: int height = 8;
   Circle bottom = new Circle(origin,radius);
   bottom.draw();
   Point p1 = new Point(-radius,0);
   Point p2 = new Point(radius,0);
   Point p3 = (Point)p1.clone();
   Point p4 = (Point)p2.clone();
   p3.move(0,height);
   p4.move(0,height);
   Line l1 = new Line(p1,p3);
   Line l2 = new Line(p2,p4);
   l1.draw();
   l2.draw();
   bottom.move(0,height);
   bottom.draw();
 }
}
\end{verbatim}

%10
\item Write a {\bf complete} stand--alone(non applet) Java program which
prompts a user to enter a 7 digit integer without commas and then
prints the integer with the commas separating the millions and thousands
and the thousands and hundreds.

\begin{verbatim}
import ccj.*

public class Commas
{
  public static void main(String[] args)
  {
    System.out.print("Enter a 7 digit number:");
    int inNum = Console.in.readWord();
    String outNum = inNum.substring(0,1) + ","
               + inNum.substring(1,4) + ","
               + inNum.substring(4,7);
    System.out.print(outNum);
  }
}

//Can also be done using integers and % and /
\end{verbatim}
\end{enumerate}
\end{document}
